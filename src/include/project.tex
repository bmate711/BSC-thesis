%--------------------------------------------------------------------------------------
% Feladatkiiras (a tanszeken atveheto, kinyomtatott valtozat)
%--------------------------------------------------------------------------------------
\clearpage
\begin{center}
\large
\textbf{FELADATKIÍRÁS}\\
\end{center}

% A feladatkiírást a tanszéki adminisztrációban lehet átvenni,
% és a leadott munkába eredeti, tanszéki pecséttel ellátott és
% a tanszékvezető által aláírt lapot kell belefűzni (ezen oldal
% \emph{helyett}, ez az oldal csak útmutatás). Az elektronikusan
% feltöltött dolgozatban már nem kell beleszerkeszteni ezt a feladatkiírást.

Az egyre növekvő ipari fejlődés következménye, hogy manapság egyre elterjedtebbé
válnak a teljesen automatizált gyártósorok. Míg régen az ipari robotok alkalmazása
kizárólag a tömeggyártásban volt jellemző, manapság már a kis sorozatszámú vagy akár
az egyedi gyártásban is megjelennek. Ennek következményeként megnőtt az igény az
egyre bonyolultabb, változatosabb ipari robotok és robotos gyártócellák alkalmazására.
Ezek már nem csak egy egyszerű ipari folyamatokat hajtanak végre, hanem egyre
összetettebb feladatokat látnak el, amelyeket már nem lehet előre beállítani a gyártósor
beüzemelésekor. Ehelyett célszerű mesterséges intelligenciát használni a probléma
megoldásához.

A feladt célja egy komplex robots pakolási feladat részfolyamatainak összehangolása és integrációja.
A feladat során a hallgatónak egy olyan programot kell elkészíteni, amely a bementére kapott,
a robotkar által megfogható munkadarabok, valós térbeli koordinátáiból megállapítja a robotkar 
ütközésmentes pályáját, ami közel ideális úton végrehajtja a pakolási feladatot. 
A feladat kidolgozása a következő lépésekből áll:

\begin{itemize}
    \item A kapcsolódó szakirodalom alapján röviden tekintese át a feladattal kapcsolatos általános alapfogalmakat.
    \item Ismertesse az egyes részfolyamatok feladatát.
    \item Tervezze meg a program be- és kimeneteli állományát.
    \item Implementálja az inverz kinematikai probláma matematikai megoldását.
    \item Valósítsa meg a részfolyamatok integrációját.
    \item Demostrálja az elkészült rendszert, és mutasson be futási eredményeket.
    \item Értékelje és összegezze az elkészült programot.
\end{itemize}
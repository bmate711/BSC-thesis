\chapter{Irodalmi áttekintés nov 19}

\section{Robotok kinematikai modellezése}
\subsection{Feladat tér}
A robotikában a feladat tér egy pontját egy M  4x4-es homogén transzformációs mátrixszal adjuk meg. Ez a mátrix a robotkar rögzítési pontjához kötött koordináta rendszer és a robotkar végpontjához kötött koordináta rendszer között add meg egy transzformációt.
$$ M = 
\begin{bmatrix}
  R_{3x3} & T_{3x1}\\
  0_{1x3} & 1
\end{bmatrix}$$
Az M mátrix egy 3x1-es T részmátrixa írja le a rögzítési ponthoz kötött Descartes koordináta rendszerben, hogy hol található a végpont koordinátarendszere. Azt, hogy a végponthoz viszonyított koordinátarendszer, hogy helyezkedik el a rögzítési ponthoz képest, azaz a két koordinátarendszer megfelelő tengelyei milyen szöget zárnak be egymással,egy 3x3-as R részmátrix írja le. Ezzel az R részmátrixal lényegében a robotkar végének állását írjuk le. 

[Robotic Task Sequencing Problem: A Survey]
\subsection{Robotkonfigurációs tér}

A robotkar állását, konfigurációját megadhatjuk a robotkar egyes csuklóinak állásával. Ekkor C legyen az egyes csuklók állásának szögértéke. Egy 3 szabadságfokú robotkar esetén C három darab értéket fog tartalmazni.

A robotkarkonfigurációs teret szokás még csukló térnek vagy robot térnek nevezni.

[KÉP]

[Robotic Task Sequencing Problem: A Survey]
\subsection{Denavit–Hartenberg paraméter}

A robotkar geometriai reprezentációját többféleképpen is megtudjuk adni.
Az egyik kényelmes módja, hogy minden egyes kartaghoz egy koordinátarendszert rögzítünk. 
Jacques Denavit és Richard Hartenberg írta le először egy konvencionális jelölésrendszert, hogy általánosítsa a kartagokhoz rendelt koordinátarendszerek leírását.A jelölésrendszer elemeit az 1955-ös publikációjuk alapján Danavit-Hartenberg paramétereknek (D-H paramétereknek) nevezzük őket.

Ebben a jelölésrendszerben négy paraméterre van szükségünk, hogy a egy koordinátarendszert leírjunk egy másik koordináta rendszerhez viszonyítva. A négy paraméterből kettő vonatkozik a csuklókra és kettő a kartagok leírására. A csuklóknál a hozzájuk kapcsolódó kartagok eltolása d és a csuklóállás szöge $\Theta$ . A kartagoknál a kartag hossza a és a kartag elcsavarodása $\alpha$ .

A jelölésrendszebren az i-edik csukló az i-edik kartag végén helyezkedik el, az i-edik és i+1-edik  kartag között. A csukló eltolás d$_i$ és a csuklószög $\Theta_i$ az i-1.csukló szerint van mérve, ezért a csukló indexe és a csukló paramétereinek indexe nem egyezik meg.

[KÉP]



[Hivatkozás: Springer Handbook of Robotics 2.6
WIkipedia: $https://en.wikipedia.org/wiki/Denavit-Hartenberg_parameters$]


\subsection{Forward kinematika}

A forward kinematika feladata nyílt kinematikai láncú robotkar esetén, hogy a kar csuklóállásainak értékéből, egy viszonyítási ponthoz tekintve, meghatározza a robotkar végpontjának helyzetét. A viszonyítási pont általában a robotkar rögzítési pontja, ami egy konstans eltolással megkapható a 0. kartagból. Általában a robotkar végére valamilyen eszközt szerelünk, hogy befolyásolni, vagy érzékelni tudja a környezetét. A robotkar végpontjának, tool frame-jének, a felszerelt eszköz végpontját szokás hívni. A tool frame megkapható egy konstans transzformációval a robot utolsó kartagjához képest.

A számításhoz bemeneti paraméterként két dologra van szükség, egyik a robotkar fizikai paraméterei, a másik a robotkar csuklóállásainak értéke.

A forward kinematikai probléma megoldása egy transzformáció egy a robotkar rögzítési pontjához kötött koordináta rendszerből a tool-framehez kötött koordináta rendszerbe, azaz a felszerelt eszköz és a rögzítési pont között. Ezt a transzformációt egy  4x4-es homogén transzformációs mátrixszal szokás leírni. A transzformáció egyértelmű, mert egy robotkarkonfigurációhoz egy homogén transzformációs mátrixot rendel. 

[Hivatkozás: Springer Handbook of Robotics 2.6]

\subsection{Inverz kinematika}
\section{Sorrendtervezés}
\subsection{GTSP}
\subsection{LNS}
\subsection{Megoldó algoritmusok}
\subsubsection{Lokális keresés}

Az informatikában a lokális keresés egy heurisztikus módszer nagy számításigényű optimalizációs feladatok megoldására. A lokális keresés segítségével megtalálhatjuk a legjobb megoldást egy probléma megoldásainak halmazából. Egy lokális kereső algoritmus a megoldási térben, megoldásról megoldásra halad, lokális változásokat végrehajtva. Az algoritmusok addig futnak, amíg el nem érnek egy olyan megoldáshoz, amin már nem tudnak tovább javítani vagy amíg a keresés ideje le nem telik.

Lokális keresést nagyon sok féle nagy számítási igényű feladaton alkalmaznak, beleértve az utazó ügynök problémát is.
Optimalizációs problémák megoldása során a lokális keresés egy célfüggvény segítségével keresi meg a megoldások közül a legjobbat. Ha az állapotteret és a célfüggvényt egy diagramon ábrázoljuk, akkor a feladat állapottérfelszínét kapjuk. Egy teljes lokális keresés mindig talál megoldást, ha az létezik. Az optimális lokális keresési algoritmusok mindig megtalálják a keresett globális minimumot vagy maximumot.


\subsubsection{Hegymászó keresés}

A hegymászó keresés egy egyszerű mohó algoritmus, ami mindig az aktuális legjobb érték felé lép. Ha az algoritmus a következő lépésben már nem tud javítani, akkor a keresés megáll. A hegymászó keresés nem optimális lokális keresés. Mivel a keresés nem járja be a teljes állapotteret, csak amíg az eredményen javítani tud, a keresés könnyen lokális optimumban ragadhat.

\subsubsection{Szimulált lehűtés}


A szimulált lehűtés a hegymászó algoritmus továbbfejlesztett változata. A hegymászó algoritmussal a legnagyobb probléma, hogy könnyen egy lokális optimumba juthatunk és befejeződik a keresés. Ez abból adódik, hogy a keresés során az állapottérnek csak minimális részét járjuk be. A szimulált lehűtés ezt próbálja meg kiküszöbölni. A hegymászó algoritmus mindig a legjobb lépést választja, ehelyett a szimulált lehűtés egy véletlen lépést választ. Ha a lépés javítja a célfüggvény értékét, akkor mindig végrehajtásra kerül. Ellenkező esetben az algoritmus a lépést csak P eséllyel teszi meg. A P valamilyen 1-nél kisebb szám, értéke exponenciálisan csökken a lépés értékének rosszaságával, azaz egy delta E mennyiséggel, amivel a célfüggvény értéke romlott. A valószínűség a szimulált lehűtés másik T paraméterétől is függ. A T hőmérséklet csökkentésével is csökken a rossz lépés valószínűsége. A rossz lépések esélye a szimulált lehűtés indulásánál, amikor a T nagy, elég magas, a T csökkenésével viszont egyre valószínűtlenebbekké válnak. Ha a T értékét kellően lassan csökkentjük akkor matematikailag bebizonyítható, hogy a szimulált lehűtés közel egy valószínűséggel megtalálja a keresett globális optimumot.

Az algoritmus a 80-as évek elején terjedt el. A szimulált lehűtést először a VLSI nyomtatott áramkörök tervezésekor használták. Azóta széleskörben alkalmazzák a nagy számításigényű optimalizációs feladatokra.

\subsubsection{Tabu keresés}



A tabu keresés lényegében egy rövidtávú memóriával rendelkező hegymászó keresés, pár szabállyal kiegészítve. A tabu keresés során addig folytatjuk a keresést amíg ki nem elégítjük a keresés feltételét, ez lehet egy célfüggvényérték vagy a futási idő korlátozása. A tabu keresési algoritmus, ha lehet mindig olyan lépést választ, amivel javít a célfüggvény értékén. Ha nincs ilyen lehetőség akkor rontó lépést is elfogad. Az algoritmus egy véges nagyságú FIFO memóriában tárolja el a már meglátogatott csomópontokat. Ha egy csomópont benne van a memóriájában akkor oda nem lép az algoritmus. Az algoritmus képes arra, hogy kitörjön lokális optimumból. Ennek hatékonysága a FIFO méretétől függ, addig elkerüli az algoritmus a lokális optimumot amíg a csomópont a memóriában van.

A tabu keresés algoritmust a 1989 években írta le Fred W. Glover. A keresési eljárást azóta széleskörben alkalmazták a logisztika, orvosbiológiai elemzés, ütemezés és egyéb nagy számítási teljesítményű optimalizációs feladatnál.

\section{Pályatervezés}
